\documentclass[12pt]{article}
\usepackage[a4paper,margin=2cm]{geometry}

\usepackage{lmodern}
\usepackage{url}
\renewcommand{\labelenumi}{\theenumi)}
\usepackage{multicol}

\title{\Large Proposed Translations for the IOL 2006 Team Contest\\\large Problem 1. American Sign Language}
\author{KUMAGAE Yuki\\
{\small\url{kumagae.yuki@iolingjapan.org}}}
\date{20240725}

\begin{document}

\pagestyle{empty}

\begin{center}
\textbf{Fourth International Olympiad in Theoretical, Mathematical and Applied Linguistics}

\textit{Tartu (Estonia), 1--6 August 2006}

Team Contest Problem
\end{center}

Given are video recordings of utterances in ASL\footnote{ASL (American Sign Language) is a sign language used by people with hearing and speech impairment in the USA, some regions of Canada and Mexico as well as some other countries. According to different estimates, from 500 thousand to several million people use ASL.} (American Sign Language) and translations of these utterances into English:

\begin{multicols}{2}
\begin{enumerate}
    \item John shoots Frank.
    \item John loves Mary.
    \item John will finish seeing Mary.
    \item John gives the girl boxes.
    \item Who likes John?
    \item John knows who Mary loves.
    \item Who does John love?
    \item I saw John yesterday, him.
    \item John will buy a car, he.
    \item John buys a house.
    \item John does not buy a house.
    \item John gives Mary a book.
    \item John will not buy a house.
    \item Bill there hits Bob.
    \item Those girls give John boxes.
    \item What, what does John like?
    \item ``Who told Bill yesterday?''\\``Mary''
    \item Who did John lipread yesterday?
    \item A student has the videotape.
    \item John carelessly writes his homework.
    \item John gives a woman over there a book.
\end{enumerate}
\end{multicols}

\noindent\underline{Assignment 1:} Translate into English the utterances contained in video recordings \url{30_task1.mov}--\url{37_task8.mov}. Explain your solution.

\bigskip

\noindent\underline{Assignment 2:} Describe how the meaning of `carelessly' is expressed in phrase 20.

\bigskip

\noindent\underline{Assignment 3:} Compile as complete a description of the grammar and vocabulary of ASL as possible.

\bigskip

\noindent\underline{Note:}

ASL (contrary to popular belief) is not a gestural form of English, but a completely separate language.
To solve the problem, knowledge of English more than is necessary to understand the sentences given in the condition is not required.

ASL users often perform gestures not quite distinctly, which is natural in rapid speech.
Many gestures have variants: some can be performed with one hand or two, while their meaning does not change.

\bigskip

Problem author: Aleksandrs Berdicevskis

Translator (tentative): Kumagae Yuki

\begin{center}
    \textit{\textbf{We wish you success!}}
\end{center}
\end{document}
