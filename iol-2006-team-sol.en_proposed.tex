\documentclass[12pt]{article}
\usepackage[a4paper,margin=2cm]{geometry}

\usepackage{lmodern}
\usepackage{url}
\renewcommand{\labelenumi}{\theenumi)}

\title{\Large Proposed Translations for the IOL 2006 Team Contest Answers\\\large Problem 1. American Sign Language}
\author{KUMAGAE Yuki\\
{\small\url{kumagae.yuki@iolingjapan.org}}}
\date{20240725}

\begin{document}

\pagestyle{empty}

\begin{center}
\textbf{Fourth International Olympiad in Theoretical, Mathematical and Applied Linguistics}

\textit{Tartu (Estonia), 1--6 August 2006}

Team Contest Answers\footnote{Courtesy of Aleksandrs Berdicevskis: \url{https://elementy.ru/problems/1982/Amerikanskiy_zhestovyy_yazyk} (Adapted by Kumagae Yuki)}
\end{center}

\noindent\underline{Assignment 1:}
Establishing vocabulary translations would be straightforward if ASL speakers performed signs slowly and carefully.
Instead, as noted, signs are, firstly, reduced, and secondly, often merge with each other to varying degrees.
Compare, for example, how the sign `to buy' is performed in the dictionaries (\url{https://www.signasl.org/}, \url{https://www.signingsavvy.com/}, \url{https://www.handspeak.com/word/}, etc.) and in phrase 9 (\textit{will buy a car}).
Additionally, some signs (for example, \textit{to love}) are sometimes performed with two hands, sometimes with one.

On the other hand, many signs possess a property called \textbf{iconicity} in linguistics:
they resemble what they mean, and their meaning is easy to guess (for example, \textit{car}, \textit{house}, \textit{to hit}, \textit{boxes}, \textit{to give}, \textit{to shoot}, etc.).

The meaning of some signs can be guessed, although they are not iconic in the strict sense.
For instance, it is hard to claim that the action `to love' resembles bringing two fists to the chest, but we understand that this sign points to the chest, where the heart is located, with which we are thought to love.
The sign for \textit{to say}, in turn, starts at the mouth (however, it can probably be considered iconic:
the word seems to fly out of the mouth into the air).

Names like \textit{John} and \textit{Mary} are conveyed in ASL using fingerspelling (manual alphabet):
speakers simply show them letter by letter, with symbols sometimes resembling the corresponding English letters, but not always.

Regarding grammar, we need to establish how verb tense, negation, questions, and plural are expressed.

For the future tense, there is a special sign (palm to cheek), corresponding to the English \textit{will}.
Past tense is not specifically expressed;
we understand that it refers to the past from context:
for example, if the sentence contains the word \textit{yesterday} (for the expression of time in the languages of the world, see IOL 2005 Individual Contest Problem 1 (\url{https://ioling.org/problems/2005/i1/})).

Negation is expressed by a separate sign (forward movement with an extended thumb), which is accompanied by shaking the head, often also with a corresponding facial expression.

A question is indicated by the presence of an interrogative word (\textit{who} or \textit{what}).
The sentence \textit{likes John who} is translated as `Who likes John?', and the sentence \textit{John love who?} as `Who does John love?', meaning who is the subject and who is the object is determined by word order.
In example 16, the interrogative pronoun stands at both the beginning and end of the sentence, which is reflected in the translation by its repetition. In an indirect question (example 6: \textit{John knows who Mary loves}), the word order is the same as in English.
It is believed that the boundaries of subordinate clauses are often denoted by non-manual markers:
the speaker slightly tilts their head back, raises their eyebrows and upper lip\footnote{Ryan Tweney, Scott Liddell \& Ursula Bellugi. 1983. The perception of grammatical boundaries in sign language \textit{// Discourse processes} 6(3): 295--304..}:
perhaps in this example we see precisely this phenomenon.

Plural can be expressed in different ways. In example 15, we understand that there are many girls, not one, because the speaker performs the sign for the demonstrative pronoun (\textit{those}) not simply by pointing to the side, but by describing a semicircle with their finger, as if pointing to many girls.
They do something similar at the end of the example with boxes.
In example 4, however, number is expressed differently, and finding the number indicator is not easy.
Nevertheless, if you carefully compare the sign for `boxes' in 4 and 15, you can see that in 15 the sign ends with depicting the sides of the box, while in 4, after this, the speaker brings together horizontally positioned palms turned upward.
This last sign is not actually part of the sign for \textit{box}, but means `many' and is often used to express plural.

\begin{enumerate}\setcounter{enumi}{29}
    \item What did John buy yesterday?
    \item Mary loves John.
    \item John will buy a house.
    \item John tells Mary, ``I am buying a house.''
    (The pronoun \textit{I} did not appear in the condition; one needs to guess that pointing to oneself indicates precisely this pronoun.
    There were also no examples of direct speech yet, but it is difficult to find another reasonable interpretation for the observed signs.)
    \item John is giving those girls boxes.
    (This time the plural is expressed not by a continuous semicircle, but by multiple repetitions of the pronoun (but this is even simpler).)
    \item Who loves John?
    \item That girl is giving John boxes.
    \item That student has not finished the homework.
\end{enumerate}

\bigskip

\noindent\underline{Assignment 2:}
With an appropriate facial expression.
Non-manual markers (facial expressions, lip movements, facial movements, body and head positions) generally play a considerable role in sign languages.

\bigskip

\end{document}
